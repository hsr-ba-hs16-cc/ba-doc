\chapter{Abstract}

Unter dem Begriff \gls{apt} versteht man gezielte und langfristig geplante Cyber Attacken, so wie es das \gls{eda} und die RUAG im Jahre 2016 erlebten. Wenn eine \gls{apt} Attacke identifiziert wird, dann stehen zeitraubende Analysearbeiten an, während dessen der Angriff weiter fortschreitet. Es stellt sich die Frage, wie der Angriff verhindert oder zumindest verzögert werden kann, ohne dass die Täterschaft dieses Eingreifen bemerkt.\\

Diese Bachelor Arbeit stellt eine Methodik und Tool vor, wie der Schaden im Unternehmen reduziert und gleichzeitig die Chancen für ein Unerkannt bleiben bei der Täterschaft erhöht werden kann. Im Wesentlichen handelt es sich um ein Verfahren um Zeit zu gewinnen, damit die \gls{apt} Attacke im Detail untersucht werden kann ohne weiteren Schaden zu riskieren.\\

Das Resultat der Arbeit umfasst eine Software, die es erlaubt, durch Trojaner (Malware) infizierte Clients im Unternehmensnetzwerk zu erkennen und stillzulegen. Die Malware ist aus Sicht des Angreifers funktionsfähig, in der Tat ist die Malware jedoch im Sleep-Mode, ohne dass dies vom Angreifer bemerkt wird. Somit gewinnt das betroffene Unternehmen Zeit, um den Fall im Detail zu klären und die Gefahr, die durch die Malware für das Unternehmen entsteht, abzuwenden.