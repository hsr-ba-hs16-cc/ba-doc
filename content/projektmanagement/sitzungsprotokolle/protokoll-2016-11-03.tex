\documentclass[class=scrbook,crop=false]{standalone}

\input{./preamble}

\begin{document}

	\section{Projektsitzung 03. November 2016}

	\begin{tabular}{ll}
		\textbf{Datum} & 03.11.2016 \\
		\textbf{Zeit} & 9:00\textendash10:00 Uhr \\
		\textbf{Ort} & Raum 1.223, HSR, Oberseestrasse 10, 8640 Rapperswil-Jona \\
		\textbf{Anwesende} & Andreas Rinkel \\ & Daniel Frei \\ & \ibuf \\ & \fbif \\ & \sadf
	\end{tabular}

	\subsection*{Traktanden}
	\begin{itemize}
		\item Weiteres Vorgehen
		\item Sprint 3: Trojaner mit Verschlüsselung?
		\item Aufbau der Arbeit (Inhaltsverzeichnis)
		\item Wissenschaftlich beleuchtet?
	\end{itemize}

	\subsection*{Was wurde erreicht}
	\begin{itemize}
		\item Triggerfish Agent Prototyp
		\item Dokumentation auf dem neusten Stand
		\item Testumgebung einsatzbereit
		\item Erste Tests
		\item Prototyping soweit abgeschlossen
	\end{itemize}

	\subsection*{Beschlüsse}
	\begin{itemize}
		\item Stand der Technik: genauer Beschreiben
		\item Auch dokumentieren was man lernen musste (z.B. ElasticSearch)
		\item Kontext: Genauere Beschreibung für was die Software ist (Zeitgewinnung, Kein Analysetool)
		\item Herr Rinkel und Herr Frei sind bereit die Arbeit vorher anzuschauen. (Stichprobe, ein Kapitel)
		\item Weiteres Ziel: Enterprise tauglich
		\item Keine zu komplexen Diagramme in der Präsentation (Kein UML SSD)
	\end{itemize}

	\subsubsection*{Weiteres Vorgehen}
	\begin{itemize}
		\item Das System erweitern: Entschlüsselung + Kampagne + evtl. Design der Software verbessern
		\item Tests mit Trojaner
	\end{itemize}

\end{document}
