\documentclass[class=scrbook,crop=false]{standalone}

\input{./preamble}

\begin{document}

	\section{Projektsitzung 30. September 2016}

	\begin{tabular}{ll}
		\textbf{Datum} & 30.09.2016 \\
		\textbf{Zeit} & 9:00\textendash10:00 Uhr \\
		\textbf{Ort} & Compass Security, Werkstrasse 20, 8645 Jona SG \\
		\textbf{Anwesende} & \ibuf \\ & \fbif \\ & \sadf
	\end{tabular}

	\subsection*{Traktanden}
	\begin{itemize}
		\item Fake C\&C Varianten
		\item IP Adressen und oder Domain Namen der C\&C Server bekannt?
		\item C\&C Server blocken vs. Paket vom Trojaner abfangen
	\end{itemize}

	\subsection*{Was wurde erreicht}
	\begin{itemize}
		\item Anforderungen für Prototyp festgelegt
		\item Proxyanalyse
		\item Kleiner Prototyp programmiert
	\end{itemize}

	\subsection*{Beschlüsse}
	\begin{itemize}
		\item Variante 2 nicht möglich (Mix aus beiden Varianten)
    \item Sitzungen mit Präsentation beginnen (roter Faden für die Sitzungen)
		\item Aufgabenstellung wird zugestellt
	\end{itemize}

	\subsubsection*{Weiteres Vorgehen}
	\begin{itemize}
    	  \item Sprint 1 fortsetzen
				\item Variante 1 und 2 zusammenführen
				\item Bessere Lösung finden als mitmproxy
				\item Analyse um Requests direkt an Fake C\&C weiter zu senden (reset 3 Way Handshake)
	\end{itemize}

\end{document}
