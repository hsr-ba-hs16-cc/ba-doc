\documentclass[class=scrbook,crop=false]{standalone}

\usepackage{../../../style}
% Name and Full Name Macros
\newcommand{\sad}{S.\ Adrian\xspace}
\newcommand{\sadf}{Silvan Adrian\xspace}

\newcommand{\fbi}{F.\ Binna\xspace}
\newcommand{\fbif}{Fabian Binna\xspace}

%Betreuer
\newcommand{\ibuf}{Ivan Bütler\xspace}
\newcommand{\ibu}{I.\ Bütler\xspace}

%Gegenleser
\newcommand{\arif}{Prof. Dr. Andreas Rinkel\xspace}

%Experte SwissRE
\newcommand{\daff}{Daniel Frei\xspace}


%-------------------------------------
%
% User Stories and Epic Styles
%
%-------------------------------------
\newtcolorbox{epic}[2][]
{
  colframe = epic!50!black!90,
  colback  = epic!40,
  colbacktitle = epic!80,
  title    = #2,
  #1,
}


\newtcolorbox{story}[2][]
{
  colframe = story!50!black!90,
  colback  = story!40,
  colbacktitle = story!80,
  title    = #2,
  #1,
}


% Remove section numbering
\renewcommand*\thesection{}

% Remove spacing between section numbering and section titl
\makeatletter
\renewcommand*{\@seccntformat}[1]{\csname the#1\endcsname}
\makeatother

\begin{document}

	\section{Projektsitzung 21. November 2016}

	\begin{tabular}{ll}
		\textbf{Datum} & 21.11.2016 \\
		\textbf{Zeit} & 11:00\textendash12:00 Uhr \\
		\textbf{Ort} & Raum 1.212a, HSR, Oberseestrasse 10, 8640 Rapperswil-Jona \\
		\textbf{Anwesende} & \ibuf \\ & \fbif \\ & \sadf
	\end{tabular}

	\subsection*{Traktanden}
	\begin{itemize}
		\item Inhalt der Abgabe festlegen (was noch alles rein?)
		\item Softwaredokumentation? Anhang? oder als Part, wie jetzt?
		\item Testumgebung Konfiguration ändern? oder dann doch Docker Umgebung aufbauen? -> Enterprise Tauglichkeit
		\item Trojaner AES
		\item Enterprise Tauglichkeit?
		\item Zentrale DB für Campaigns
	\end{itemize}

	\subsection*{Was wurde erreicht}
	\begin{itemize}
		\Item Erster Trojaner mit Testumgebung getestet
		\item Triggerfish Campaigns
		\item Triggerfish Threading
		\item Pufferfish XOR Decryption
		\item Pufferfish Threading
		\item Fake C\&C Server
		\item Dokumentation auf dem neusten Stand
	\end{itemize}

	\subsection*{Beschlüsse}
	\begin{itemize}
		\item Keine Enterprise Tauglichkeit mehr, SSL Bump Problem (In Doku erwähnen)
		\item Nächstes Meeting 2.12.16 9:00 Compass Security
	\end{itemize}

	\subsubsection*{Weiteres Vorgehen}
	\begin{itemize}
		\item Git Repository, Campaign verteilen
		\item AES 128
		\item Elasticsearch Jobs, Decryption
		\item Ports für Fake C&C (Um mehrere Fake C&Cs auf gleicher VM zu hosten)
	\end{itemize}

\end{document}
