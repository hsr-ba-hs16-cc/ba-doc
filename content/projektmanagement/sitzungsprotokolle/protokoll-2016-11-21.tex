\documentclass[class=scrbook,crop=false]{standalone}

\input{./preamble}

\begin{document}

	\section{Projektsitzung 21. November 2016}

	\begin{tabular}{ll}
		\textbf{Datum} & 21.11.2016 \\
		\textbf{Zeit} & 11:00\textendash12:00 Uhr \\
		\textbf{Ort} & Raum 1.212a, HSR, Oberseestrasse 10, 8640 Rapperswil-Jona \\
		\textbf{Anwesende} & \ibuf \\ & \fbif \\ & \sadf
	\end{tabular}

	\subsection*{Traktanden}
	\begin{itemize}
		\item Inhalt der Abgabe festlegen (was noch alles rein?)
		\item Softwaredokumentation? Anhang? oder als Part, wie jetzt?
		\item Testumgebung Konfiguration ändern? oder dann doch Docker Umgebung aufbauen? -> Enterprise Tauglichkeit
		\item Trojaner AES
		\item Enterprise Tauglichkeit?
		\item Zentrale DB für Campaigns
	\end{itemize}

	\subsection*{Was wurde erreicht}
	\begin{itemize}
		\Item Erster Trojaner mit Testumgebung getestet
		\item Triggerfish Campaigns
		\item Triggerfish Threading
		\item Pufferfish XOR Decryption
		\item Pufferfish Threading
		\item Fake C\&C Server
		\item Dokumentation auf dem neusten Stand
	\end{itemize}

	\subsection*{Beschlüsse}
	\begin{itemize}
		\item Keine Enterprise Tauglichkeit mehr, SSL Bump Problem (In Doku erwähnen)
		\item Nächstes Meeting 2.12.16 9:00 Compass Security
	\end{itemize}

	\subsubsection*{Weiteres Vorgehen}
	\begin{itemize}
		\item Git Repository, Campaign verteilen
		\item AES 128
		\item Elasticsearch Jobs, Decryption
		\item Ports für Fake C&C (Um mehrere Fake C&Cs auf gleicher VM zu hosten)
	\end{itemize}

\end{document}
