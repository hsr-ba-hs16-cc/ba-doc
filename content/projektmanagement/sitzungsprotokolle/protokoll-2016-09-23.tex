\documentclass[class=scrbook,crop=false]{standalone}

\input{./preamble}

\begin{document}

	\section{Projektsitzung 23. September 2016}

	\begin{tabular}{ll}
		\textbf{Datum} & 23.09.2016 \\
		\textbf{Zeit} & 8:00\textendash9:00 Uhr \\
		\textbf{Ort} & Compass Security, Werkstrasse 20, 8645 Jona SG \\
		\textbf{Anwesende} & \ibuf \\ & \fbif \\ & \sadf
	\end{tabular}

	\subsection*{Traktanden}
	\begin{itemize}
		\item Abgrenzung der Arbeit: Nur Erkennung C\&C?
		\item Lösungsansätze: Data Mining? Machine Learning? \\
			Nur Analyse des Netzwerkverkehrs erlaubt bzw. nur Proxy
		\item Ziel der Arbeit? Umfang der Abgabe? Proxy Software, Analyse?
		\item Testumgebung? Vorhanden, selber aufsetzen?
		\item Testen der Lösung? Diverse Trojaner? Diverse Testumgebungen? Netzwerkaufzeichnungen zum testen?
		\item Zugriff auf Repository, Jira ...
		\item Termine für weitere Besprechungen (Sprints)? Intervall?
		\item Private Repositories?
		\item Bewertungsbogen?
		\item Nächste Schritte
	\end{itemize}

	\subsection*{Was wurde erreicht}
	\begin{itemize}
		\item Aufsetzen der Dokumentation
		\item Einarbeiten ins Thema APT
		\item Aufsetzen der CI Umgebung
	\end{itemize}

	\subsection*{Beschlüsse}
	\begin{itemize}
    	\item keine public Repositories (Minimum: Lösung des Problems, Programcode des Fake C\&C)
    	\item Nächster Termin 30.09.16 9:00 - 10:00 Uhr
    	
	\end{itemize}

	\subsubsection*{Weiteres Vorgehen}
	\begin{itemize}
    	  \item Sprint 1 planen
    	  \item Eigener Client-Server mit Fake C\&C (Zu Demonstrationszwecken)
    	  \item Umbau Infrastruktur (Private Repositories)
	\end{itemize}

\end{document}
