\documentclass[class=scrbook,crop=false]{standalone}

\input{./preamble}

\begin{document}

	\section{Projektsitzung 14. Oktober 2016}

	\begin{tabular}{ll}
		\textbf{Datum} & 14.10.2016 \\
		\textbf{Zeit} & 9:00\textendash10:00 Uhr \\
		\textbf{Ort} & Compass Security, Werkstrasse 20, 8645 Jona SG \\
		\textbf{Anwesende} & \ibuf \\ & \fbif \\ & \sadf
	\end{tabular}

	\subsection*{Traktanden}
	\begin{itemize}
		\item Präventive oder nachträgliche Abwehr von Trojaner -> siehe Beschlüsse
		\item Iptables Prototyp (Erste Vorschau auf Möglichkeiten)
		\item Logik beim Proxy, integrierbarkeit mit bestehenden Proxies?
		\item Squid Proxy (weitverbreiteter Open Source Proxy -> besser auf bestehende Proxies setzen als selbst einen programmieren)
	\end{itemize}

	\subsection*{Was wurde erreicht}
	\begin{itemize}
		\item FiddlerCore Prototyp
		\item Analyse zu Möglichkeiten
	\end{itemize}

	\subsection*{Beschlüsse}
	\begin{itemize}
		\item Nachträgliche Abwehr des Trojaners
		\item Die Software wird Fish Agent genannt
		\item Termin 21.10.16 9:00 - 10:00
		\item Termin 03.11.16 9:00 - 10:00
	\end{itemize}

	\subsubsection*{Weiteres Vorgehen}
	\begin{itemize}
		\item Prototyp V2 erstellen
		\item Proxy mit ElasticSearch (filebeat, topbeat, packetbeat)
		\item Redirect mit Iptables
	\end{itemize}

\end{document}
