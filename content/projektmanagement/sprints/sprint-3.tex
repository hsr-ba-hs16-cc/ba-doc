\subsection{Sprint 3}

\subsubsection*{Summary}

\begin{table}[H]
	\centering
	\begin{tabular}{ll}
		\toprule
		\multicolumn{2}{c}{\textbf{Sprint 3}}\\
		\midrule
		\textbf{Periode} & 31.10.2016 \textendash 20.11.2016\\
		\textbf{Stunden Soll} & \SI{168}{\hour}\\
		\textbf{Story Points} & 79\\
		\textbf{Stunden Ist} & \SI{182,5}{\hour}\\
		\bottomrule
	\end{tabular}
\end{table}

\subsubsection*{Ziele}
\begin{itemize}
  \item Release Candidate 1
  \item Dokumentation verbessern
  \item Technische Dokumentation
\end{itemize}


\subsubsection*{Abgeschlossen}
Folgende User Stories wurden während dem dritten Sprint abgeschlossen.
\begin{table}[H]
    \centering
	\begin{tabularx}{\textwidth}{l X l}
	\toprule
	\textbf{JIRA-Key} & \textbf{Summary} & \textbf{Story Points}\\
	\midrule
      \textbf{CC-37} & Als Entwickler will ich eine aktuelle Dokumentation. & 20 \\
      \textbf{CC-38} & Als Security Engineer will ich eine Campaign definieren. & 5\\
      \textbf{CC-39} & Als Entwickler will ich chunked Data speichern können. & 8\\
      \textbf{CC-40} & Als Entwickler will ich eine performante Analyse der Log Daten. & 20 \\
      \textbf{CC-41} & Als Security Engineer will ich verschlüsselten Payload analysieren können. & 5 \\
      \textbf{CC-42} & Als Entwickler will ich erkennen, ob der Payload verschlüsselt ist. & 8\\
      \textbf{CC-43} & Als Entwickler will ich einen Fake C\&C zum Testen. & 13\\
	\bottomrule
    \end{tabularx}
    \caption{Projektmanagement: Sprint 3 User Stories}
\end{table}

\subsubsection*{Probleme}
-
