\subsection{Sprint 0}

\subsubsection*{Summary}

\begin{table}[H]
	\centering
	\begin{tabular}{ll}
		\toprule
		\multicolumn{2}{c}{\textbf{Sprint 0} \textit{(Kickoff)}}\\
		\midrule
		\textbf{Periode} & 19.09.2016\textendash 25.09.2016\\
		\textbf{Stunden Soll} & \SI{57}{\hour}\\
		\textbf{Stunden Plan} & \SI{55.5}{\hour} \\
		\textbf{Stunden Ist} & \SI{49.75}{\hour}\\
		\bottomrule
	\end{tabular}
\end{table}


\subsubsection*{Ziele}
\begin{itemize}
  \item Infrastruktur/Entwicklungsumgebung aufsetzen
  \item Aufgabenstellung verstehen
  \item Erste Analyse zu C\&C
\end{itemize}


\subsubsection*{Abgeschlossen}
Folgende Tasks wurden während dem ersten Sprint abgeschlossen.
\begin{table}[H]
    \centering
	\begin{tabularx}{\textwidth}{l X l}
	\toprule
	\textbf{JIRA-Key} & \textbf{Summary}\\
	\midrule
        \textbf{CC-1} & Dokumentation aufsetzen \\
	\textbf{CC-2} & Infrastruktur/Entwicklungsumgebung aufsetzen \\
	\textbf{CC-3} & Kickoffmeeting 1. Woche \\
	\textbf{CC-4} & Erste Analyse zu C\&C, Advance Persistence Threat\\
	\textbf{CC-5} & Grobe Projektplanung \\
	\bottomrule
    \end{tabularx}
    \caption{Projektmanagement: Sprint 0 Tasks}
\end{table}

\subsubsection*{Probleme}
Unklarheiten bei Inhalt und Struktur der Dokumentation.