\newpage
\section{Sprints}
\subimport{}{sprint-0.tex}
\newpage
\subimport{}{sprint-1.tex}
\newpage
\subimport{}{sprint-2.tex}
\newpage
\subimport{}{sprint-3.tex}
\newpage
\subimport{}{sprint-4.tex}
\newpage
\subimport{}{abschluss.tex}
\newpage
\section{Stundenübersicht und Fazit}

Für die BA wurden im ganzen \underline{\SI{870}{\hour}} benötigt und \underline{\SI{720}{\hour}} sind der Mindestaufwand (1 ECTS an 30 Stunden).


\subsection{Personen}

\textbf{Silvan Adrian} \underline{\SI{425}{\hour}} \\
\textbf{Fabian Binna} \underline{\SI{445}{\hour}}

\subsection{Kategorien}

\textbf{Dokumentation} \underline{\SI{440}{\hour}}\\
\textbf{Analyse Allgemein} \underline{\SI{67}{\hour}}\\
\textbf{Analyse Malware} \underline{\SI{75}{\hour}}\\
\textbf{Entwicklung} \underline{\SI{266}{\hour}}\\
\textbf{Allgemein} \underline{\SI{22}{\hour}}\\

\subsection{Fazit}

Der Aufwand der Sprints wurde durch die User Stories eingeschätzt. Am Ende kann man die Genauigkeit der Schätzung nachvollziehen.

\begin{table}[H]
    \centering
	\begin{tabularx}{\textwidth}{l X c c}
       \toprule
        \textbf{Sprint} & \textbf{Story Points} & \textbf{Stunden Ist} & \textbf{Anz. Stunden/Story Point}\\ \hline
  	  \midrule
      \textbf{Sprint 1} & 45 & \SI{113.75}{\hour} & \SI{2.53}{\hour} \\
      \textbf{Sprint 2} & 86 & \SI{174.5}{\hour} & \SI{2.03}{\hour}\\
      \textbf{Sprint 3} & 79 & \SI{182.5}{\hour} & \SI{2.31}{\hour}\\
      \textbf{Sprint 4} & 94 & \SI{223}{\hour} & \SI{2.37}{\hour}\\
	\bottomrule  
    \end{tabularx}
    \caption{Projektmanagement: Schätzungsgenauigkeit}
\end{table}

Abweichungen sind hier nur im Zehntel Bereich, und gegen Ende der Arbeit wurden wir bereits besser mit der Genauigkeit.

\subsubsection{Restliche Zeit}
Die restliche Zeit brauchten wir für die Vorbereitungen (Sprint 0) und den Abschluss (nach Ende des 4. Sprints), wobei die Dokumentation am meisten Aufwand benötigte.