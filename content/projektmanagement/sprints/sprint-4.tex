\subsection{Sprint 4}

\subsubsection*{Summary}

\begin{table}[H]
	\centering
	\begin{tabular}{ll}
		\toprule
		\multicolumn{2}{c}{\textbf{Sprint 4}}\\
		\midrule
		\textbf{Periode} & 21.11.2016 \textendash 11.12.2016\\
		\textbf{Stunden Soll} & \SI{168}{\hour}\\
		\textbf{Story Points} & 94\\
		\textbf{Stunden Ist} & \SI{223}{\hour}\\
		\bottomrule
	\end{tabular}
\end{table}

\subsubsection*{Ziele}
\begin{itemize}
  \item Release Candidate 2
  \item Dokumentation verbessern
\end{itemize}


\subsubsection*{Abgeschlossen}
Folgende User Stories wurden während dem dritten Sprint abgeschlossen.
\begin{table}[H]
    \centering
	\begin{tabularx}{\textwidth}{l X l}
       \toprule
        \textbf{JIRA-Key} & \textbf{Beschreibung} & \textbf{Story Points}\\ \hline
  	  \midrule
      \textbf{CC-46} & 	Als Entwickler will ich eine aktuelle Dokumentation. & 40 \\
      \textbf{CC-47} & Als Security Engineer will ich alle Systeme über einen zentralen Punkt verwalten. & 20\\
      \textbf{CC-48} & Als Auftraggeber will ich Payload mit \gls{aes128} entschlüsseln können. & 13\\
      \textbf{CC-49} & Als Entwickler will ich mehrere Fake C\&C betreiben & 8 \\
      \textbf{CC-50} & Als Entwickler will ich auf der Elasticsearch entschlüsseln können. & 13 \\
	\bottomrule  
    \end{tabularx}
    \caption{Projektmanagement: Sprint 4 User Stories}
\end{table}

\subsubsection*{Probleme}
Arbeiten, die nicht eingeplant waren, wurden auch noch in diesen Sprint genommen (Zusätzliche Strategy), da aber in diesem Sprint mit hohem Aufwand gerechnet wurde (94 Story Points), konnten die Folgen davon ein bisschen abgeschwächt werden.
Eine User Story existiert aber trotzdem nicht.
