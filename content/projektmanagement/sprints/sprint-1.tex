\subsection{Sprint 1}

\subsubsection*{Summary}

\begin{table}[H]
	\centering
	\begin{tabular}{ll}
		\toprule
		\multicolumn{2}{c}{\textbf{Sprint 1}}\\
		\midrule
		\textbf{Periode} & 26.09.2016\textendash 09.10.2016\\
		\textbf{Stunden Soll} & \SI{112}{\hour}\\
		\textbf{Story Points} & 45\\
		\textbf{Stunden Ist} & \SI{113,75}{\hour}\\
		\bottomrule
	\end{tabular}
\end{table}

\subsubsection*{Ziele}
\begin{itemize}
  \item Prototyp
  \item Machbarkeitsanalyse
  \item Proxyanalyse
  \item Entwicklungsumgebung
\end{itemize}


\subsubsection*{Abgeschlossen}
Folgende User Stories wurden während dem zweiten Sprint abgeschlossen.
\begin{table}[H]
    \centering
	\begin{tabularx}{\textwidth}{l X l}
	\toprule
	\textbf{JIRA-Key} & \textbf{Summary} & \textbf{Story Points}\\
	\midrule
        \textbf{CC-6} & Als Entwickler will ich eine Entwicklungsumgebung & 3 \\
         \textbf{CC-7} & Als Auftraggeber will ich einen Prototyp & 8\\
          \textbf{CC-8} & Als Entwickler will ich eine Dokumentation zum Prototyp & 3\\
          \textbf{CC-9} & Als Auftraggeber will ich eine Machbarkeitsanalyse & 13\\
          \textbf{CC-10} & Als Entwickler will ich eine Proxyanalyse & 13 \\
          \textbf{CC-12} & Als Entwickler will ich eine aktuelle Projektdokumentation & 5 \\
	\bottomrule
    \end{tabularx}
    \caption{Projektmanagement: Sprint 1 User Stories}
\end{table}

\subsubsection*{Probleme}
Es ergaben sich hauptsächlich Probleme mit den programmierten Prototypen, welche 
nicht perfekt funktionieren wollten und so relativ viel Zeit benötigten.
