\glsaddall
%Akronyme
\newacronym{hsr}{HSR}{Hochschule für Technik Rapperswil}
\newacronym{ba}{BA}{Bachelorarbeit}
\newacronym{icap}{ICAP}{Internet Content Adaptation Protocol}
\newacronym{fts}{FTS}{Full Text Search}
\newacronym{apt}{APT}{Advanced Persistent Threat}
\newacronym{eda}{EDA}{Eidgenössische Departement für auswärtige Angelegenheiten}
\newacronym{ssl}{SSL}{Secure Socket Layer}
\newacronym{rest}{REST}{Representational State Transfer}
\newacronym{api}{API}{Application Programming Interface}
\newacronym{acl}{ACL}{Access Control List}
\newacronym{cc}{C\&C}{Command \& Control}
\newacronym{ca}{CA}{Certification Authority}
\newacronym{sni}{SNI}{Server Name Indication}
\newacronym{ide}{IDE}{Integrated Development Environment}


%Glossary

\newglossaryentry{0day}{name=0-Day, description={Bei einem 0-Day handelt es sich um einen Bug der seit Release einer Software oder nach einem Update vorhanden ist.}}

\newglossaryentry{aes128}{name=AES128, description={Beim Advanced Encryption Standard 128 handelt es sich um eine symmetrische Verschlüsselung mit einer Schlüssellänge von 128 Bit.}}

\newglossaryentry{san}{name=SAN, description={Mit den Subject Alternative Names können mehrere Hostnames einem Zertifikat hinterlegt werden.}}

\newglossaryentry{cn}{name=CN, description={Unter dem Common Name wird ein Zertifikat ausgestellt (Hostname).}}

\newglossaryentry{rc4}{name=RC4, description={Bei Arcfour handelt es sich um eine Symmetrische Verschlüsselung}}

\newglossaryentry{scm}{name=SCM, description={Als Software-Configuration-Management verwenden wir Git zur Versionsverwaltung der Dokumentation und des Source Codes}}


